\documentclass{article}

\usepackage{fullpage}
\usepackage{amsmath}
\usepackage{latexsym}
\usepackage{algorithm}
\usepackage{algpseudocode}
\usepackage{dsfont}
\usepackage{mathtools}
\usepackage[standard]{ntheorem}
\usepackage{hyperref}
\hypersetup{
    bookmarks=true,         % show bookmarks bar?
    unicode=false,          % non-Latin characters in AcrobatÕs bookmarks
    pdftoolbar=true,        % show AcrobatÕs toolbar?
    pdfmenubar=true,        % show AcrobatÕs menu?
    pdffitwindow=false,     % window fit to page when opened
    pdfstartview={FitH},    % fits the width of the page to the window
    pdftitle={My title},    % title
    pdfauthor={Author},     % author
    pdfsubject={Subject},   % subject of the document
    pdfcreator={Creator},   % creator of the document
    pdfproducer={Producer}, % producer of the document
    pdfkeywords={keyword1} {key2} {key3}, % list of keywords
    pdfnewwindow=true,      % links in new window
    colorlinks=true,       % false: boxed links; true: colored links
    linkcolor=red,          % color of internal links (change box color with linkbordercolor)
    citecolor=blue,        % color of links to bibliography
    filecolor=magenta,      % color of file links
    urlcolor=cyan           % color of external links
}
\usepackage{times}
\usepackage{natbib}
\usepackage[capitalize]{cleveref}

\title{This is a title}
\author{These are authors}

\begin{document}
\maketitle

\section{Introduction}

Useful citations are \cite{KW14,GKK13}.

\section{Algorithm}

\begin{algorithm}[H]
\caption{Connect nodes}\label{alg:cool}
\begin{algorithmic}
\State Let $X = set of all nodes$
\For{$x \in X$}
\For{$y \in x \ldots X_n$}
\If{$dist(x,y) < \gamma \And x.type \ne y.type$}
\State connect(x,y)
\EndIf
\EndFor
\EndFor
\end{algorithmic}
\end{algorithm}

This is a reference to \cref{alg:cool}

You can do equations too
\begin{align}
3 + 5 
&= 8 \\
&= 16 - 8 \\
&= 32 - 16 - 8
\end{align}
or with functions
\begin{align}
\sqrt{\frac{1}{n} \log n} + \frac{c}{n} + \gamma \exp\left(-\frac{1}{d}\right)
\end{align}

\begin{theorem}
This theorem is true.
\end{theorem}

\begin{proof}
A proof that the previous theorem is true.
\end{proof}

\subsection{A subsection}

\subsubsection{A subsubsection}

\subsubsection*{An unnumbered subsubsection} blah blah

\section{Experimental Results}

\section{Conclusion}

\bibliographystyle{plain}
\bibliography{nn}

\end{document}

